\documentclass[a4paper,lang=cn,a4paper]{elegantpaper}
\title{Relativistic Quantum Theory of Atoms and Molecules}

\date{}
\usepackage{mhchem}
\usepackage{siunitx}
\usepackage{bm}
\usepackage{url}
\usepackage{amsfonts}
\newcommand{\field}[1]{\mathbb{#1}}
\newcommand{\vect}[1]{\boldsymbol{#1}}
\begin{document}
\maketitle
\section{Relativity in atomic and molecular physics}

\subsection{Elementary ideas}
nuclei $\longrightarrow$ point mass \hspace{3cm}
a-th nucleus $\longrightarrow$ $Z_a e$

The moving particles interact according to Coulomb's law:
\begin{align*}
    -&\frac{Z_e e^2}{4\pi \varepsilon_0 r^2} & \text{nucleus a -- electrons}\\[1em]
    & \frac{e^2}{4\pi \varepsilon_0 r^2} & \text{electron -- electron}\\[1em]
    & \frac{Z_a Z_b e^2}{4\pi \varepsilon_0 r^2} & \text{nuclei a--b repel each other}
\end{align*}

The electronic intrinsic angular momentum -- spin $\vect{s}$
\begin{equation*}
    \vect{s}=\frac{1}{2}\hbar\vect{\sigma} \qquad \vect{\sigma}=(\sigma_x, \sigma_y, \sigma_z)
\end{equation*}
Note: $\vect{s}^2$ and $s_z$ have eigenfunction. $\sigma -- spin label $

N indistinguishable electrons system wavefunction $\Psi(q_1, q_2,\dotsc q_N,t)$

Spin-Orbit Coupling $\longrightarrow \vect{j}=\vect{l}+\vect{s}$

\subsection{The one-electron atom}
\subsubsection{Classical Kepler orbits}
\begin{equation*}
    \frac{1}{r}=\frac{mk}{|\vect{l}|^2}\{1+\varepsilon\cos(\theta+\alpha)\} \qquad
    k=\frac{Ze^2}{4\pi\varepsilon_0} \qquad \varepsilon=\sqrt{1+\frac{2E|\vect{L}|^2}{mk^2}}
\end{equation*}

\begin{itemize}
    \item $-\frac{mk}{2|\vect{l}|^2}\leqslant E <0 \quad \Longrightarrow \quad 0\leqslant\varepsilon<1$
    \begin{equation*}
        \Longrightarrow \left\{
        \begin{aligned}
            r &= \frac{|\vect{l}|^2}{mk(1+\varepsilon)} &\text{closest approach}\\
            r &= \frac{|\vect{l}|^2}{mk(1-\varepsilon)} &\text{maximum distance}
        \end{aligned}\right.
    \end{equation*}
    When $\varepsilon = 0 \quad \Longrightarrow \quad r=\frac{l^2}{mk}\quad \Longrightarrow \quad E=-\frac{mk}{2|\vect{l}|^2}$.
    \item $E=0\quad \Longrightarrow \quad \varepsilon=1 \quad \Longrightarrow \quad$ orbit is a parabola.
    \item $E>0\quad \Longrightarrow \quad \varepsilon>1 \quad \Longrightarrow \quad$ orbit is hyperbola.
\end{itemize}
\begin{equation*}
    r_{min}=\frac{|\vect{l}|^2}{mk(1+varepsilon)} \qquad v_{max}=\frac{|\vect{l}|}{mr_{min}}
\end{equation*}

\subsubsection{The Bohr atom}
\begin{equation*}
    E = \frac{1}{2} \left\langle V\right\rangle = -\left\langle T\right\rangle
\end{equation*}
Where $E$ is the energy of particle, $\left\langle T \right\rangle$
is the orbital average of the kinetic energy and 
$\left\langle V\right\rangle$ is the potential energy.

The frequencies of the spectral lines could be fitted to Rydberg's 
formula: 
\begin{equation*}
    \nu = R\left(\frac{1}{n^2}-\frac{1}{m^2}\right)
\end{equation*}
The transition energy between two states: $E_n = -\frac{R}{n^2}$

\subsubsection{X-ray spectra and Moseley's Law}
The square root of the frequency of each corresponding X-ray line 
was approximately proportional to Z. Relativistic effects modify the 
Z-dependence as Z increases.

\subsubsection{Transition to quantum mechanics}
A particle wavefunction: $\psi(\vect{r},t)$ 

Schr\"{o}dinger equation: 
$i\hbar\frac{\partial \psi}{\partial t}\psi(\vect{r},t)=H\psi(\vect{r},t)$

Hamiltonian: $\hat{H}(\vect{p}, \vect{r})=\frac{1}{2m}\vect{p}^2+V(\vect{r}),\quad \vect{p}\rightarrow -i\hbar\Delta,\quad \vect{r}\rightarrow \vect{r}$

$V(\vect{r})\rightarrow$ potential energy of an electron at a distance $r=|\vect{r}|$
\begin{equation*}
    V(\vect{r})=-\frac{Ze^2}{4\pi \varepsilon r}
\end{equation*}
which could deduce the Hamiltonian
\begin{equation*}
    \hat{H}=-\frac{\hbar^2}{2m}\Delta^2 - \frac{Ze^2}{4\pi \varepsilon r}
\end{equation*}
and whoes energies are given by the formula
\begin{equation*}
    \varepsilon_{nl}=-\frac{mZ^2 e^4}{32\pi^2 \varepsilon^2_0 \hbar^2 n^2}
\end{equation*}

The orbital angular momentum vector $\vect{l}$, and $\vect{l}^2$ takes 
the values $l(l+1)\hbar^2$, $l_z$ takes the $2l+1$ values $m\hbar$.
Due to Rydberg's formula $R=mZ^2e^4/32\pi^2\varepsilon^2_0 \hbar^2$, 
from the energy relation could deduce that $\left\langle T_n\right\rangle=-E_n$,
and $T_n=mv^2/2$, get the relation
\begin{equation*}
    \frac{v_n}{c}=\frac{\alpha Z}{n}
\end{equation*}
where $\alpha=e^2/4\pi\varepsilon_0\hbar c$ is the dimensionless
fine structure constant. 

In spherical polar coordinates
\begin{equation*}
    \psi_{nlm}(\vect{r},t)=\text{const.}\frac{P_{nl}(r)}{r}Y_l^m(\theta,\phi)
\end{equation*}

\subsubsection{Sommerfeld's relativistic orbits and Dirac's wave equation}
In the Kepler problem, the particle speed attains its maximum at closest
approach to the centre of force
\begin{equation*}
    \frac{v_{max}}{c}=\frac{k}{c|\vect{l}|}(1+\varepsilon),\qquad \varepsilon>0
\end{equation*}
$v_{max}$ is inversely proportional to $|\vect{l}|$, so the largest 
effects in states with the lowest angular momentum.

A particle moving in some reference frame with velocity $\vect{u}$ has
four-momentum $p^u$ 
\begin{equation*}
    p^0=\frac{E}{c}=mc\gamma(u), \quad p^i =mu^i \gamma(u),\quad i=1,2,3.
\end{equation*}
where $\gamma(u)=1/\sqrt{1-u^2/c^2}$ and $u^i$ are the Cartesian components 
of $\vect{u}$.

Dirac' relativistic wave equation for hydrogenic atoms:
\begin{equation*}
    i\hbar\frac{\partial \psi}{\partial t}=\hat{H}\psi,\qquad 
    \hat{H}=c\vect{\alpha}\cdot\vect{p}+\beta mc^2-\frac{Ze^2}{4 \pi \varepsilon_0 r}
\end{equation*}

\end{document}