\documentclass[a4paper,lang=cn,a4paper]{elegantpaper}
\title{Relativistic Quantum Theory of Atoms and Molecules}

\date{}
\usepackage{mhchem}
\usepackage{siunitx}
\usepackage{bm}
\usepackage{url}
\usepackage{amsfonts}
\newcommand{\field}[1]{\mathbb{#1}}
\newcommand{\vect}[1]{\boldsymbol{#1}}
\begin{document}
\maketitle
\section{Relativity in atomic and molecular physics}

\subsection{Elementary ideas}
nuclei $\longrightarrow$ point mass \hspace{3cm}
a-th nucleus $\longrightarrow$ $Z_a e$

The moving particles interact according to Coulomb's law:
\begin{align*}
    -&\frac{Z_e e^2}{4\pi \varepsilon_0 r^2} & \text{nucleus a -- electrons}\\[1em]
    & \frac{e^2}{4\pi \varepsilon_0 r^2} & \text{electron -- electron}\\[1em]
    & \frac{Z_a Z_b e^2}{4\pi \varepsilon_0 r^2} & \text{nuclei a--b repel each other}
\end{align*}

The electronic intrinsic angular momentum -- spin $\vect{s}$
\begin{equation*}
    \vect{s}=\frac{1}{2}\hbar\vect{\sigma} \qquad \vect{\sigma}=(\sigma_x, \sigma_y, \sigma_z)
\end{equation*}
Note: $\vect{s}^2$ and $s_z$ have eigenfunction. $\sigma -- spin label $

N indistinguishable electrons system wavefunction $\Psi(q_1, q_2,\dotsc q_N,t)$

Spin-Orbit Coupling $\longrightarrow \vect{j}=\vect{l}+\vect{s}$

\subsection{The one-electron atom}
\subsubsection{Classical Kepler orbits}
\begin{equation*}
    \frac{1}{r}=\frac{mk}{|\vect{l}|^2}\{1+\varepsilon\cos(\theta+\alpha)\} \qquad
    k=\frac{Ze^2}{4\pi\varepsilon_0} \qquad \varepsilon=\sqrt{1+\frac{2E|\vect{L}|^2}{mk^2}}
\end{equation*}

\begin{itemize}
    \item $-\frac{mk}{2|\vect{l}|^2}\leqslant E <0 \quad \Longrightarrow \quad 0\leqslant\varepsilon<1$
    \begin{equation*}
        \Longrightarrow \left\{
        \begin{aligned}
            r &= \frac{|\vect{l}|^2}{mk(1+\varepsilon)} &\text{closest approach}\\
            r &= \frac{|\vect{l}|^2}{mk(1-\varepsilon)} &\text{maximum distance}
        \end{aligned}\right.
    \end{equation*}
    When $\varepsilon = 0 \quad \Longrightarrow \quad r=\frac{l^2}{mk}\quad \Longrightarrow \quad E=-\frac{mk}{2|\vect{l}|^2}$.
    \item $E=0\quad \Longrightarrow \quad \varepsilon=1 \quad \Longrightarrow \quad$ orbit is a parabola.
    \item $E>0\quad \Longrightarrow \quad \varepsilon>1 \quad \Longrightarrow \quad$ orbit is hyperbola.
\end{itemize}
\begin{equation*}
    r_{min}=\frac{|\vect{l}|^2}{mk(1+varepsilon)} \qquad v_{max}=\frac{|\vect{l}|}{mr_{min}}
\end{equation*}

\subsubsection{The Bohr atom}
\begin{equation*}
    E = \frac{1}{2} \left\langle V\right\rangle = -\left\langle T\right\rangle
\end{equation*}
Where $E$ is the energy of particle, $\left\langle T \right\rangle$
is the orbital average of the kinetic energy and 
$\left\langle V\right\rangle$ is the potential energy.

The frequencies of the spectral lines could be fitted to Rydberg's 
formula: 
\begin{equation*}
    \nu = R\left(\frac{1}{n^2}-\frac{1}{m^2}\right)
\end{equation*}
The transition energy between two states: $E_n = -\frac{R}{n^2}$

\subsubsection{X-ray spectra and Moseley's Law}
The square root of the frequency of each corresponding X-ray line 
was approximately proportional to Z. Relativistic effects modify the 
Z-dependence as Z increases.

\subsubsection{Transition to quantum mechanics}





\end{document}